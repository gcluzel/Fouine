\documentclass[12pt]{beamer}
\usepackage[utf8]{inputenc}
\usepackage[frenchb]{babel}
\usepackage[T1]{fontenc}
\usepackage{amsmath}
\usepackage{amsfonts}
\usepackage{amssymb}
\usepackage{graphicx}
\author{Guillaume \textsc{Cluzel} et Julien \textsc{Devevey}}
\title{Projet Fouine (Intermédiaire)}

\hypersetup{pdfpagemode=FullScreen} 

%\setbeamercovered{transparent} 
%\setbeamertemplate{navigation symbols}{} 
%\logo{} 
%\institute{} 
%\date{} 
%\subject{} 
\begin{document}

\begin{frame}
\titlepage
\end{frame}

%\begin{frame}
%\tableofcontents
%\end{frame}

\begin{frame}[fragile]{les références}

  \begin{itemize}
  \item Gestion différente des références plus proches de ce qu'on pourrait faire en OCaml.
  \end{itemize}

A la place de :

\begin{verbatim}
    let a = ref 0 in
        let b = (a := 1) in
        !b
\end{verbatim}

On écrira :

\begin{verbatim}
    let a = ref 0 in
        let b = (a := 1; !a) in
        !b
\end{verbatim}


\end{frame}





\begin{frame}[fragile]{Les exceptions}
\begin{itemize}
\item On a créé une exception \texttt{E}
\begin{verbatim}
    let y = try
               raise (E 0)
            with E 1 -> prInt 15
\end{verbatim}

\item Ce code retourne une erreur \texttt{E 0}.

\item Malheuresement on ne peut pas attraper d'autres erreurs que celles déclenchées par \texttt{raise}.
\end{itemize}

\end{frame}






\begin{frame}{L'interpréteur mixte}

L'interpréteur mixte fonctionne de la manière suivante :


\vspace{12pt}  

\begin{itemize}
  \item Pour une expression arithmétique contenant uniquement des variables, on utilise la machine à pile.
  
    \vspace{12pt}  
  
  \item Pour l'évaluation de fonctions, les références et les exceptions, on bascule sur l'interpréteur.
\end{itemize}

\end{frame}



\begin{frame}{Conclusion}
\begin{itemize}
\item Ce qui a été plus difficile :
\begin{itemize}
\item L'interprétation et  l'évaluation des fonctions et les portée des variables.
\item Dans le parser le plus dur a été de parser l'application de fonction
\end{itemize}

\item Sinon tout fonctionne correctement.
\end{itemize}
\end{frame}


\end{document}